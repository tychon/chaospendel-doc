\documentclass[a4paper, 10pt]{article}
\usepackage[utf8]{inputenc}
\usepackage[T1]{fontenc}
\usepackage{ngerman, lmodern}
%\usepackage[left=3cm, right=3cm]{geometry}

\newcommand{\mytitle}{Untersuchung und Kontrolle von chaotischem Verhalten am Doppelpendel}
\newcommand{\myauthor}{Jann Horn und Hannes Riechert}

\usepackage{graphicx, wrapfig, subfigure, color, amsmath, amsthm}
\usepackage{fancyhdr}
  \lhead{\myauthor}
  \chead{}
  \rhead{\nouppercase\leftmark}
  \lfoot{}
  \cfoot{\thepage}
  \rfoot{}

\usepackage[numbers]{natbib}
\usepackage{setspace}
  \onehalfspacing % Zeilenabstand
% \usepackage[numbers]{natbib} % bibliography

\newcommand{\TODO}{\textcolor{red}{ \textbf TODO }}
\newcommand{\mathematik}{\begin{equation*}\begin{aligned}}
\newcommand{\mathematikstop}{\end{aligned}\end{equation*}}
\renewcommand{\phi}{\varphi} % make "stroked" phi look "loopy"
\renewcommand{\rho}{\varrho}
\newcommand{\half}{\frac{1}{2}} % 1/2
\newcommand{\phid}{\dot{\phi}}  % phi with one dot
\newcommand{\intend}{\,\mathrm{d}} % end of integral
\newcommand{\EQU}{\qquad\bigg|\,} % separator for explanations for equivalent rearrangements

%\clubpenalty = 10000
%\widowpenalty = 10000
%\displaywidowpenalty = 10000

\title{\mytitle}
\author{\myauthor}
%\date{ }

\begin{document}
\maketitle
\begin{abstract}
In unserem Projekt beschäftigen wir uns mit dem Verhalten von chaotischen Doppelpendeln. Wir wollen aus der aktuellen Bewegung eines Doppelpendels den weiteren Bewegungsablauf in einem kurzen Zeitintervall extrapolieren und dann versuchen, diese Bewegung zu beeinflussen.

Hierzu wollen wir zunächst ein Doppelpendel konstruieren, bei dem Daten über den aktuellen Bewegungszustand erfasst werden können. Diese Daten sollen in Echtzeit von einem Computer ausgewertet werden, um laufend eine Prognose an die Messwerte anzupassen. Anhand dieser Prognose soll dann entschieden werden, ob das Pendel eine unerwünschte Bewegung durchführen wird, und wenn nötig, soll mithilfe mehrerer Spulen eine korrigierende magnetische Kraft erzeugt werden. Es könnte zum Beispiel erwünscht sein, einen Überschlag zu vermeiden.
\end{abstract}

\newpage
\pagestyle{fancy}
\tableofcontents

\newpage

\section{Einleitung}
Als erstes haben wir an den mathematisch genauen Schwingungsdifferenzialgleichungen für das Doppelpendel gearbeitet.
Und eine vollkommen theoretische Simulation von einem Doppelpendel realisiert.

Erst danach haben wir mit der Konstruktion des Modells, das in Abbildung (\TODO ref) zu sehen ist begonnen.

\section{Simulation}

\subsection{Mathematischer Hintergrund}
Der Lösungsansatz über die Euler"=Lagrange Gleichung ist von der englischen Wikipedia übernommen worden. \citep{wikidoublependulum}

Die Gleichungen wurden nur um die Konstanten $k_1$ bis $k_5$ erweitert, um sie an unser Pendel anpassen zu können, das keine gleichmäßige Massenverteilung aufweist.

(\TODO wo ist die vollständige Herleitung?)

Die Position der Pendelarme kann durch die zwei generalisierten Koordinaten $\phi_1$ und $\phi_2$ beschrieben werden, die die Winkel der Arme angeben.
Für die vollständige Angabe des Zustands des Systems ist zusätzlich noch der generalisierte Impuls hier als $p_1$ und $p_2$ bezeichnet, vonnöten.

Aus den Funktionen $\dot{p}_1$, $\dot{p}_2$, $\phid_1$ und $\phid_2$ setzt sich das System von Differentialgleichungen zusammen, mit denen aus dem aktuellen Zustand der Zustand des Systems zu jeder gegebenen Zeit berechnen lässt.
Die Berechnung kann zum Beispiel Iterativ mit dem Runge"=Kutta"=Verfahren erfolgen.

(\TODO bild um variablen zu erklären)

\mathematik
\phid_1 &= \frac{k_4 p_1 - k_5 l_1 p_2 cos(\phi_1 - \phi_2)}{k_1 k_4 + l_1 \cdot (k_3 k_4 - k_5^2 l_1 cos^2(\phi_1 - \phi_2))} \\[0.5\baselineskip]
\phid_2 &= \frac{k_1 p_2 + l1 \cdot (k_3 p_2 - k_5 p_1 cos(\phi_1 - \phi_2))}{k_1 k_4 + l_1 \cdot (k_3 k_4 - k_5^2 l_1 cos^2(\phi_1 - \phi_2))} \\[0.5\baselineskip]
\dot{p}_1 &= -l_1 \phid_1 \phid_2 k_5 sin(\phi_1 - \phi_2) - g k_2 sin \phi_1 - g l_1 k_3 sin \phi_1 \\[0.5\baselineskip]
\dot{p}_2 &= l_1 \phid_1 \phid_2 k_5 sin(\phi_1 - \phi_2) - g k_5 sin \phi_2 \\[0.5\baselineskip]
\mathematikstop
\mathematik
k_1 &= \int^{l_{1b}}_{l_{1a}} \rho_1(r) \; r^2 \intend r
\qquad && k_2 &= \int^{l_{1b}}_{l_{1a}} \rho_1(r) \; r \intend r \\
k_3 &= \int^{l_{2b}}_{l_{2a}} \rho_2(r) \intend r
&& k_4 &= \int^{l_{2b}}_{l_{2a}} \rho_2(r) \; r^2 \intend r \\
k_5 &= \int^{l_{2b}}_{l_{2a}} \rho_2(r) \; r \intend r \\
\mathematikstop

\subsection{Software}

\section{Der Aufbau des Modells}

\begin{figure}
  \includegraphics[width=\textwidth]{charts/pendulumsketch.png}
  \caption{Position der Spulen in Polarkoordinaten}
  \label{fig:pendulumsketch}
\end{figure}

Um das Verhalten des Doppelpendels zu beeinflussen, muss der Zustand des Pendels
erfasst werden können. Zu diesem Zweck verwenden wir einen am Ende des Pendels
befestigten Magneten, der in am Rahmen montierten Spulen elektrische Ströme
hervorruft, wenn er an diesen vorbeibewegt wird.

Diese Ströme verursachen Spannungsschwankungen an den Spulen. Um diese zu messen,
verbinden wir alle Spulen auf einer Seite mit einer 3,3 V--Span"-nungs"-quel"-le und auf der
anderen mit den 16 Analogeingängen eines ATmega2560--Mi"-kro"-con"-trollers. Der auf einen
Messbereich von 0 bis 5V eingestellte Mikrocontroller iteriert dann in einer Endlosschleife
über alle Eingänge, misst die an ihnen momentan anliegenden Spannungen mit einer
Auflösung von 10 Bit und leitet sie an einen Serial--to--USB--Konverter weiter. Der Messbereich
ist dadurch zwar relativ zur Grundspannung asymmetrisch --- der maximale negative Ausschlag beträgt
theoretisch 3,3 V, während der maximale positive Ausschlag $ 5V - 3,3 V = 1.7 V $ beträgt ---, aber
Spannungsregler für 3,3 V sind einfacher verfügbar als welche für 2,5 V.
% zusätzliches argument: negative spitzen machen mist?

Der Serial--to--USB--Konverter ist per USB an einen Computer angeschlossen, der die 5V--Stromversorgung
bereitstellt und die Messwerte aufzeichnen und verarbeiten kann.

Die 10--Bit--Auflösung des Mikrocontrollers bedeutet, dass die Schrittweite für die Digitalisierung
der Eingangswerte $ 5 V / 2^{10} = 4,883 mV $ beträgt. Da allerdings bei der Messung ein
Grundrauschen mit einer Breite von ca. XXXX V auftritt, ist die Genauigkeit der Messwerte geringer.

Aus Gründen der Einfachheit --- Atmel--Mikrocontroller mit USB--Schnitt"-stel"-le sind nur als SMD--Chips
verfügbar ---, haben wir uns dazu entschlossen, ein Arduino--Mega--2560--Board zu nutzen, das einen
ATmega2560--Mikrocontroller, einen ATmegaXXU2--Mikrocontroller als Serial--to--USB--Converter und eine 
3,3 V--Span"-nungs"-quel"-le enthält. Das Programm für den ATmega2560, das die Messwerte von den
Analogeingängen liest und über eine serielle Schnittstelle an den ATmegaXXU2 weitergibt, haben
wir aber selbst in C geschrieben, anstatt die Arduino--spezifische Sprache zu verwenden, die auf
einer höheren Ebene operiert, um möglichst gute Kontrolle über die Geschwindigkeit des Programms
zu haben.

Die Programmierung auf einer solch niedrigen Ebene und das Lesen der Mikrocontroller--Dokumentation
haben gezeigt, dass 500000 Baud die ideale Datenübertragungsrate für die Kommunikation mit dem PC
sind, da bei diesem Wert der Frequenzteiler für die serielle Datenübertragung genau 1 wird.
Tests mit dem in C geschriebenen Programm für den Mikrocontroller haben gezeigt, dass bei
dieser Baudrate die
parallel zur seriellen Datenübertragung ablaufende Analog--Digital--Wandlung die langsamste
Komponente ist, weshalb es ohne Leistungseinbußen möglich ist, für die Übertragung
der Messwerte an den PC ein Protokoll mit hoher Redundanz zu verwenden.

Das Protokoll zwischen Mikrocontroller und PC überträgt Messwerte als Kombination aus
Eingangs--ID und Messwert mit hoher Redundanz, sodass fehlerhafte oder
unerwartete Werte abgefangen und
ignoriert werden können. Solche fehlerhaften Werte sind zwar im laufenden Betrieb relativ selten,
treten aber während der Initialisierung des Arduino häufig auf. Ein weiterer Zweck der
mitgesendeten Eingangs--ID ist es, ohne bidirektionale Kommunikation eine Resynchronisation des
Empfängerprogramms auf dem PC beispielsweise nach einem Neustart zu ermöglichen.


\section{Messwertverarbeitung}
\begin{figure}
  \includegraphics[width=\textwidth]{charts/network_dia.png}
  \caption{Flussdiagramm der Messdaten}
  \label{fig:network}
\end{figure}

\clearpage
\addcontentsline{toc}{section}{Literatur}
\bibliographystyle{alphadin}
\bibliography{Dokumentation}{}

\end{document}

