
\section{Der Aufbau des Modells}
Das fertige Modell ist in Abbildung \TODO zu sehen.
Das eigentliche Pendel ist an einer Dreieckskonstruktion aus Aluminiumprofilen befestigt.
Die breite Grundfläche sichert den Stand des Pendels seitlich, da in die Richtungen oft große Kräfte wirken.
Falls die Konstruktion dann verrutschen oder kippen sollte, ginge ein Teil der Energie aus dem System verloren und die Vorhersage würde beeinträchtigt werden.

(\TODO Pendelmaße)

\begin{figure}[bht]
  \includegraphics[width=\textwidth]{charts/pendulumsketch.png}
  \caption{Position der Messspulen in Polarkoordinaten}
  \label{fig:pendulumsketch}
\end{figure}

Auf der Rückseite des Aufbaus ist eine Holzplatte angebracht, auf der 17 Messspulen befestigt sind, deren Konfiguration in Abbildung \ref{fig:pendulumsketch} zu sehen ist.
Dort sind Spulen mit einer Windungszahl von 1600 Windungen rot und Spulen mit 20.000 Windungen grün markiert.

Im Schaltplan in Abbildung \ref{fig:circuit} ist zu sehen, wie die Messspulen $S_0$ bis $S_{15}$ an die analogen Eingänge des Arduino angeschlossen werden.
Anders als die roten Spulen, die ohne Widerstand auskommen, sind die grünen Spulen mit jeweils einem Vorwiderstand von $R_0 = 3,3 k\Omega$ parallel geschaltet, um die höhere erzeugte Spannung auszugleichen, da sie mehr Windungen haben.

Die zwei Arbeitsspulen $S16$ sind direkt gegenüber der untersten beiden Messspulen angebracht und haben jeweils 300 Windungen.
Sie sind für vier Ampere Strom ausgelegt, der über \TODO ?Relais ? MOSFET? geschaltet wird.
Weil unsere Spannungsquelle für einen Ampere bei $24 V$ ausgelegt ist, haben wir einen Vorwiderstand von $R16 = 22 k\Omega$ in Reihe zu den Spulen geschaltet und so den Strom begrenzt.
Zusätzlich ist zu jeder Arbeistspule eine LED mit Vorwiderstand $R_{17} = 1,8 k\Omega$ parallel geschaltet, um einfacher erkennen zu können, ob die Spule eingeschaltet ist. In Abbildung \TODO sind oben die \TODO MOSFETs? und LEDs auf dem Steckboard zu erkennen.

\begin{figure}[bht]
  \includegraphics[width=\textwidth]{charts/circuit_dia.png}
  \caption{Schaltplan für das Modell}
  \label{fig:circuit}
\end{figure}

\section{Messwertverarbeitung}

\subsection{Im Arduino}

Um das Verhalten des Doppelpendels zu beeinflussen, muss der Zustand des Pendels
erfasst werden können. Zu diesem Zweck verwenden wir einen am Ende des Pendels
befestigten Magneten, der in am Rahmen montierten Spulen elektrische Ströme
hervorruft, wenn er an diesen vorbeibewegt wird.

Diese Ströme verursachen Spannungsschwankungen an den Spulen. Um diese zu messen,
verbinden wir alle Spulen auf einer Seite mit einer 3,3 V"=Spannungsquelle und auf der
anderen mit den 16 Analogeingängen eines ATmega2560"=Mikrocontrollers. Der auf einen
Messbereich von 0 bis 5V eingestellte Mikrocontroller iteriert dann in einer Endlosschleife
über alle Eingänge, misst die an ihnen momentan anliegenden Spannungen mit einer
Auflösung von 10 Bit und leitet sie an einen Serial--to--USB--Konverter weiter. Der Messbereich
ist dadurch zwar relativ zur Grundspannung asymmetrisch --- der maximale negative Ausschlag beträgt
theoretisch 3,3 V, während der maximale positive Ausschlag $ 5V - 3,3 V = 1.7 V $ beträgt ---, aber
Spannungsregler für 3,3 V sind einfacher verfügbar als welche für 2,5 V.

Der Serial"=to"=USB"=Konverter ist per USB an einen Computer angeschlossen, der die 5V--Stromversorgung
bereitstellt und die Messwerte aufzeichnen und verarbeiten kann.

Die 10"=Bit"=Auflösung des Mikrocontrollers bedeutet, dass die Schrittweite für die Digitalisierung
der Eingangswerte $ 5 V / 2^{10} = 4,883 mV $ beträgt. Da allerdings bei der Messung ein
Grundrauschen mit einer Breite von ca. \TODO XXXX V auftritt, ist die Genauigkeit der Messwerte geringer.

Aus Gründen der Einfachheit --- Atmel--Mikrocontroller mit USB"=Schnittstelle sind nur als SMD--Chips
verfügbar ---, haben wir uns dazu entschlossen, ein Arduino"=Mega"=2560"=Board zu nutzen, das einen
ATmega2560"=Mikrocontroller, einen \TODO ATmegaXXU2"=Mikrocontroller als Serial"=to"=USB"=Converter und eine 
3,3 V"=Spannungsquelle enthält. Das Programm für den ATmega2560, das die Messwerte von den
Analogeingängen liest und über eine serielle Schnittstelle an den \TODO ATmegaXXU2 weitergibt, haben
wir aber selbst in C geschrieben, anstatt die Arduino--spezifische Sprache zu verwenden, die auf
einer höheren Ebene operiert, um möglichst gute Kontrolle über die Geschwindigkeit des Programms
zu haben.

Die Programmierung auf einer solch niedrigen Ebene und das Lesen der Mikrocontroller"=Dokumentation
haben gezeigt, dass 500000 Baud die ideale Datenübertragungsrate für die Kommunikation mit dem PC
sind, da bei diesem Wert der Frequenzteiler für die serielle Datenübertragung genau 1 wird.
Tests mit dem in C geschriebenen Programm für den Mikrocontroller haben gezeigt, dass bei
dieser Baudrate die
parallel zur seriellen Datenübertragung ablaufende Analog"=Digital"=Wandlung die langsamste
Komponente ist, weshalb es ohne Leistungseinbußen möglich ist, für die Übertragung
der Messwerte an den PC ein Protokoll mit hoher Redundanz zu verwenden.

Das Protokoll zwischen Mikrocontroller und PC überträgt Messwerte als Kombination aus
Eingangs--ID und Messwert mit hoher Redundanz, sodass fehlerhafte oder
unerwartete Werte abgefangen und
ignoriert werden können. Solche fehlerhaften Werte sind zwar im laufenden Betrieb relativ selten,
treten aber während der Initialisierung des Arduino häufig auf. Ein weiterer Zweck der
mitgesendeten Eingangs"=ID ist es, ohne bidirektionale Kommunikation eine Resynchronisation des
Empfängerprogramms auf dem PC beispielsweise nach einem Neustart zu ermöglichen.

\subsection{Auf dem Computer}

\begin{figure}
  \includegraphics[width=\textwidth]{charts/network_dia.png}
  \caption{Flussdiagramm der Messdaten}
  \label{fig:network}
\end{figure}

Wir haben ein System aus Tools programmiert, die Informationen untereinander austauschen können und jedes Programm erfüllt einen bestimmten Zweck.
Im Flussdiagramm in Abbildung \ref{fig:network} ist zu erkennen, welche Daten ausgetauscht werden.
Die durchgezogenen Pfeile sind Unix Domain Sockets, die gestrichelten Linien sind Dateien, die beim Programmstart geladen werden.

\subsubsection{ArduinoReader}
Der \app{ArduinoReader} verbindet sich über eine serielle Schnittstelle mit dem Arduino und ließt die Daten aus, die der Arduino gemessen hat.
Die Datensätze werden mit einem Zeitstempel versehen und über ein Unix Domain Socket an andere Programme weiter gesendet und in der Datei \app{ReplayFile} gespeichert.
Ein Datensatz besteht aus 16 Zahlen im Bereich von 0 bis 1023, die die Messwerte für jede Spule angeben.
Der Bereich diser Zahl wird von der 10"=Bit"=Genauigkeit des Analog"=Digital"=Wandlers vorgegeben.
Außerdem können Kommandos zum Schalten der digitalen Ausgänge an den \app{ArduinoReader} gesendet werden, die dann an den Arduino weitergeleitet werden.

\subsubsection{Replayer}
Anstatt des \app{ArduinoReader}s kann der \app{Replayer} eingesetzt werden, der vom \app{ArduinoReader} aufgezeichnete Daten abspielt.
So können mehrere Analysen von gleichen Daten gemacht werden, oder gearbeitet werden, ohne dass ein Arduino angeschlossen ist.

\subsubsection{Normalisation}
Die \app{Normalisation} liest die Daten des \app{ArduinoReader}s und berechnet für jede Spule separat Durchschnittswerte und Standardabweichung.
Zusätzlich werden noch der größte und der kleinste gemessene Wert aufgezeichnet, die das Rauschen eingrenzen.
Die Ergebnisse werden in die Datei \app{NoiseInfo} geschrieben.

Dieses Programm wird ausgeführt, wenn sich das Pendel nicht bewegt.
So kann später erkannt werden, ob Messwerte signifikant außerhalb des Rauschens liegen, und es können auch die Messwerte verschiedener Spulen verglichen werden.

\subsubsection{Rangetester}
\begin{wrapfigure}{r}{5cm}
  \includegraphics[width=0.48\textwidth]{images/rangetester.png}
  \caption{Screenshot vom \app{Rangetester}}
  \label{fig:screenrangetester}
\end{wrapfigure}
Der \app{Rangetester} zeigt zu jeder Spule den aktuellen Messwert an.
Wir haben ihn benutzt, um den besten Widerstandswert für die Spulen mit 20.000 Windungen zu finden.
Hier kann auch die Auswirkung der Arbeitsspulen auf die Messpulen beobachtet werden.
\TODO bilder

\subsubsection{Tracker}
\begin{figure}
  \centering
  \includegraphics[width=\textwidth]{images/tracker.png}
  \caption{Screenshot vom \app{Tracker}}
  \label{fig:screentracker}
\end{figure}
Der \app{Tracker} benutzt die Daten aus der \app{NoiseInfo}-Datei, um Messdaten vom \app{ArduinoReader} zu normalisieren.

\subsubsection{Vorhersage}
\begin{figure}
  \centering
  \includegraphics[width=\textwidth]{images/prediction.png}
  \caption{Screenshot von der \app{MarkovPrediction}}
  \label{fig:screenmarkov}
\end{figure}
