
\section{Der experimentelle Aufbau}

\subsection{Das Doppelpendel}

\begin{figure}[Hbt]
  \centering
  \begin{subfigure}{\textwidth}
    \centering
    \includegraphics[width=0.7\textwidth]{images/real_front.jpg}
    \caption{Vorderseite des Aufbaus}
    \label{fig:front}
  \end{subfigure}
  \begin{subfigure}{\textwidth}
    \centering
    \includegraphics[width=0.7\textwidth]{images/real_back.jpg}
    \caption{Rückseite des Aufbaus}
    \label{fig:back}
  \end{subfigure}
  \caption{Bilder vom Aufbau}
  \label{fig:aufbau}
\end{figure}

Das fertige Modell ist in Abbildung \ref{fig:aufbau} zu sehen.
Das eigentliche Pendel ist an einer Dreieckskonstruktion aus Aluminiumprofilen befestigt.
Die breite Grundfläche sichert den Stand des Pendels seitlich, da in die Richtungen oft große Kräfte wirken.
Falls die Konstruktion dann verrutschen oder kippen sollte, ginge ein Teil der Energie aus dem System verloren und die Vorhersage würde beeinträchtigt werden.

\begin{figure}[Hbt]
  \centering
  \includegraphics[width=0.7\textwidth]{charts/pendulumsketch.png}
  \caption{Position der Messspulen in Polarkoordinaten (Abbildung erzeugt mit Python-Skript aus Projektdatei.)}
  \label{fig:pendulumsketch}
\end{figure}

\subsection{Messwerterfassung mit Induktionsspulen}

Um das Verhalten des Doppelpendels zu beeinflussen, muss der Zustand des Pendels erfasst werden können.
Zu diesem Zweck verwenden wir einen am Ende des Pendels befestigten Magneten, der in am Rahmen montierten Spulen durch Induktion elektrische Spannungen hervorruft, wenn er an diesen vorbeibewegt wird.

Auf der Rückseite des Aufbaus ist eine Holzplatte angebracht, auf der 17 Messspulen befestigt sind, deren Konfiguration in Abbildung \ref{fig:pendulumsketch} zu sehen ist.
Dort sind Spulen mit einer Windungszahl von 1600 Windungen rot und Spulen mit 20.000 Windungen grün markiert.

\begin{figure}[bht]
  \centering
  \includegraphics[width=0.7\textwidth]{charts/circuit_dia.png}
  \caption{Schaltplan für das Modell}
  \label{fig:circuit}
\end{figure}

Im Schaltplan in Abbildung \ref{fig:circuit} ist zu sehen, wie die Messspulen $S_0$ bis $S_{15}$ an die analogen Eingänge des Arduino angeschlossen werden.
Anders als die roten Spulen, die ohne Widerstand auskommen, sind die grünen Spulen mit jeweils einem Widerstand von $R_0 = 3,3 k\Omega$ parallel geschaltet, um die höhere erzeugte Spannung auszugleichen, da sie mehr Windungen haben.
Ohne diese Widerstände würden sie machmal Spannungen außerhalb des Messbereichs erzeugen, was die Messelektronik durcheinander brächte.

\subsection{Arbeitsspulen zur Chaoskontrolle}

Die zwei Arbeitsspulen $S_{16}$ und $S_{17}$ sind direkt gegenüber der untersten beiden Messspulen angebracht und haben jeweils 300 Windungen.
Sie sind für Stromstärken bis $4A$ ausgelegt und werden über MOSFETs geschaltet.
Zusätzlich ist zu jeder Arbeitsspule eine LED mit Vorwiderstand $R_{16} = 1,8 k\Omega$ parallel geschaltet, um einfacher erkennen zu können, ob die Spule eingeschaltet ist.


\section{Messwerterfassung und -verarbeitung}

\subsection{Der Arduino}

Um die Spannungsschwankungen an den Spulen zu messen,
verbinden wir alle Spulen auf einer Seite mit einer 3,3 V"=Spannungsquelle und auf der
anderen mit den 16 Analogeingängen eines ATmega2560"=Mikrocontrollers. Der auf einen
Messbereich von 0 bis 5V eingestellte Mikrocontroller iteriert dann in einer Endlosschleife
über alle Eingänge, misst die an ihnen momentan anliegenden Spannungen mit einer
Auflösung von 10 Bit und leitet sie an einen Serial"=to"=USB"=Konverter weiter. Der Messbereich
ist dadurch zwar relativ zur Grundspannung asymmetrisch --- der maximale negative Ausschlag beträgt
theoretisch 3,3 V, während der maximale positive Ausschlag $ 5V - 3,3 V = 1.7 V $ beträgt ---, aber
Spannungsregler für 3,3 V sind einfacher verfügbar als solche für 2,5 V.

Der Serial"=to"=USB"=Konverter ist per USB an einen Computer angeschlossen, der die 5V--Strom"-ver"-sor"-gung
bereitstellt und die Messwerte aufzeichnen und verarbeiten kann.

Die 10"=Bit"=Auflösung des Mikrocontrollers bedeutet, dass die Schrittweite für die Digitalisierung
der Eingangswerte $ 5 V / 2^{10} = 4,883 mV $ beträgt. Da allerdings bei der Messung ein
Grundrauschen mit einer Breite von ca. 50mV auftritt, ist die Genauigkeit der Messwerte geringer.

Aus Gründen der Einfachheit --- Atmel--Mikrocontroller mit USB"=Schnittstelle sind nur als SMD--Chips
verfügbar ---, haben wir uns dazu entschlossen, ein Arduino"=Mega"=2560"=Board zu nutzen, das einen
ATmega2560"=Mikrocontroller, einen weiteren ATmega"=Mikrocontroller mit USB als Serial"=to"=USB"=Converter und eine 
3,3 V"=Spannungsquelle enthält. Das Programm für den ATmega2560, das die Messwerte von den
Analogeingängen liest und über eine serielle Schnittstelle an den Microcontroller mit USB weitergibt, haben
wir selbst in C geschrieben, um möglichst gute Kontrolle über die Geschwindigkeit des Programms
zu haben, anstatt die Arduino"=spezifische Sprache zu verwenden, die auf
einer höheren Ebene operiert und langsamer ist.

Die Programmierung auf einer solch niedrigen Ebene und das Lesen der Mikrocontroller"=Dokumentation
haben gezeigt, dass 500000 Baud die ideale Datenübertragungsrate für die Kommunikation mit dem PC
sind, da bei diesem Wert der Frequenzteiler für die serielle Datenübertragung genau 1 wird.
Tests mit dem in C geschriebenen Programm für den Mikrocontroller haben gezeigt, dass bei
dieser Baudrate die
parallel zur seriellen Datenübertragung ablaufende Analog"=Digital"=Wandlung die langsamste
Komponente ist, weshalb es ohne Leistungseinbußen möglich ist, für die Übertragung
der Messwerte an den PC ein Protokoll mit hoher Redundanz zu verwenden.

Das Protokoll zwischen Mikrocontroller und PC überträgt Messwerte als Kombination aus
Eingangs--ID und Messwert mit hoher Redundanz, sodass fehlerhafte oder
unerwartete Werte abgefangen und
ignoriert werden können. Solche fehlerhaften Werte sind zwar im laufenden Betrieb relativ selten,
treten aber während der Initialisierung des Arduino häufig auf. Ein weiterer Zweck der
mitgesendeten Eingangs"=ID ist es, ohne bidirektionale Kommunikation eine Resynchronisation des
Empfängerprogramms auf dem PC beispielsweise nach einem Neustart zu ermöglichen.

\subsection{Verarbeitung am Computer}

\begin{figure}
  \centering
  \includegraphics[width=0.8\textwidth]{charts/network_dia.png}
  \caption{Flussdiagramm der Messdaten}
  \label{fig:network}
\end{figure}

Wir haben ein System aus Tools programmiert, die Informationen untereinander austauschen können. Jedes Programm erfüllt dabei möglichst genau einen bestimmten Zweck.
Im Flussdiagramm in Abbildung \ref{fig:network} ist dargestellt, welche Daten wie ausgetauscht werden.
Die durchgezogenen Pfeile sind Unix Domain Sockets, die gestrichelten Linien sind Dateien, die beim Programmstart geladen werden.
Die Programme werden nun im Einzelnen beschrieben.

\subsubsection{ArduinoReader}

Der \app{ArduinoReader} verbindet sich über eine serielle Schnittstelle mit dem Arduino und liest die Daten aus, die der Arduino gemessen hat.
Die Datensätze werden mit einem Zeitstempel versehen und über einen Unix Domain Socket sowohl an andere Programme weiter gesendet als auch in der Datei \app{ReplayFile} gespeichert.
Ein Datensatz besteht aus 16 Zahlen im Bereich von 0 bis 1023, welche die Messwerte für jede Spule angeben.
Der Bereich dieser Zahlen wird von der 10"=Bit"=Genauigkeit des Analog"=Digital"=Wandlers vorgegeben.
Außerdem können Kommandos zum Schalten der digitalen Ausgänge an den \app{ArduinoReader} gesendet werden, die dann an den Arduino weitergeleitet werden.

\subsubsection{Replayer}
Anstatt des \app{Arduino"-Reader}s kann der \app{Re"-play"-er} eingesetzt werden, der vom \app{Arduino"-Reader} aufgezeichnete Daten abspielt.
So können mehrere Analysen von gleichen Daten gemacht werden, oder es kann gearbeitet werden, ohne dass ein Arduino angeschlossen ist.

\subsubsection{Normalisation}
Die \app{Normalisation} liest die Daten des \app{ArduinoReader}s und berechnet für jede Spule separat Durchschnittswerte und Standardabweichung.
Zusätzlich werden noch der größte und der kleinste gemessene Wert aufgezeichnet, die das Rauschen eingrenzen.
Die Ergebnisse werden in die Datei \app{NoiseInfo} geschrieben.

Dieses Programm wird ausgeführt, wenn sich das Pendel nicht bewegt.
So kann später erkannt werden, ob Messwerte signifikant außerhalb des Rauschens liegen, und es können auch die Messwerte verschiedener Spulen verglichen werden.

\subsubsection{Rangetester}
Der \app{Rangetester} zeigt zu jeder Spule den aktuellen Messwert an.
Wir haben ihn benutzt, um den besten Widerstandswert für die Spulen mit 20.000 Windungen zu finden.
Hier kann auch die Auswirkung der Arbeitsspulen auf die Messpulen beobachtet werden.

\begin{wrapfigure}{r}{0.4\textwidth}
  \includegraphics[width=0.4\textwidth]{images/rangetester-white_cropped.png}
  \caption{Screenshot vom \app{Rangetester}}
  \label{fig:screenrangetester}
\end{wrapfigure}


In Abbildung \ref{fig:screenrangetester} ist ein Screenshot der Anzeige zu sehen.
Blau ist der Bereich markiert, in dem das Rauschen liegt.
Die grünen Balken zeigen den aktuellen Wert relativ zum Durchschnittswert an.

\subsubsection{Tracker}
Der \app{Tracker} benutzt die Daten aus der \app{NoiseInfo}-Datei, um Messdaten vom \app{Arduino"-Reader} zu normalisieren und berechnet die magnetische Flussdichte zu jeder Spule.

Um einen Messwert $x$ zu normalisieren, wird als erstes der Durchschnittswert $\mu$ abgezogen und dann durch die Anzahl der Windungen $n$ geteilt.
Das Ergebnis daraus wird mit dem Verhältnis des Eigenwiderstands der Spule $R_S$ zu dem Widerstandswert $R_P$ des parallel geschalteten Widerstands multipliziert:

\mathematik
norm = \frac{x - \mu}{n} \cdot \frac{R_S}{R_P}
\mathematikstop

Der normalisierte Wert $norm$ ist jetzt für alle Spulen vergleichbar.
Aus ihm wird dann ein Integral berechnet, das als magnetische Flussdichte interpretiert werden kann.
Da die Messdaten ein Rauschen enthalten und der Analog"=Digital"=Wandler nicht perfekt linear arbeitet, konvergiert die Summe aller normalisierten Daten einer Spule schnell in den stark positiven oder stark negativen Bereich.
Um diese Konvergenz zu beherrschen, werden die Integrale auf 0 zurückgesetzt, wenn in 50  aufeinanderfolgenden Datensätzen kein Messwert außerhalb des Rauschens lag.
Sie werden auch zurückgesetzt, wenn ihr Wert kleiner 0 wird.

Neben der Normalisierung sortiert der \app{Tracker} alle Datensätze aus, die verfälscht sein könnten.
Falsch ist zum Beispiel ein Datensatz, der einen Messwert enthält, der gleich 0 oder 1023 ist. In dem Fall kann es sein, dass die Spannung an einer Spule so groß ist, dass der Analog"=Digital"=Wandler für die nächsten Messungen falsche Werte liefert und der gesamte Datensatz falsch ist.

Es werden auch Datensätze verworfen, bei denen mehrere Spulen Messwerte außerhalb des Rauschens anzeigen.
Dann ist nämlich wahrscheinlich eine Arbeitsspule umgeschaltet worden, die alle Messpulen beeinflusst.

Zusätzlich versendet der \app{Tracker} den Index der Messspule mit dem größten Integral über einen Unix Domain Socket.

\subsection{Latenzmessung}

Bei der Beeinflussung muss beachtet werden, dass die Latenz (Verzögerung) der Mess- und Steuerdaten zwischen den Mess- und Beeinflussungsspulen
und dem Computer nicht zu groß ist, denn eine zu hohe Latenz führt dazu, dass die Beeinflussung zu stark zeitversetzt zur echten Pendelbewegung
passiert.
Beim Zuführen von Energie in das System hätte dies beispielsweise zur Folge, dass die Beeinflussungsspule den Magneten erst abstößt, wenn er sich bereits ein Stück weit entfernt hat, wodurch die Beeinflussung weniger stark ausfallen würde.
Um herauszufinden, ob dies für unseren Aufbau relevant ist, haben wir eine Messung der Round"=Trip"=Time durchgeführt.

Als Round"=Trip"=Time wird die Zeit bezeichnet, die ein Signal benötigt, um von einem System zum anderen und wieder zurück zu gelangen, also in
unserem Fall vom Computer in die Beeinflussungsspule und dann über die Messspule wieder zurück in den Computer.

Aus der Samplerate von 555 Samples pro Sekunde, die am Computer gemessen wird, kann auf die Ausleserate vom ADC geschlossen werden, nämlich $110 \si{\us} / Sample$. Da der Analog"=Digital"=Wandler eine sogenannte Sample"=And"=Hold"=Schaltung benutzt, also Änderungen des Messwerts nach
Messbeginn nicht mehr den gemessenen Wert beeinflussen können, benötigt ein Eingangswert also bereits mindestens diese Zeit, um den Computer
zu erreichen.

Um die Round"=Trip"=Time zu messen, haben wir ein Programm geschrieben, das das Grundrauschen einer Messspule misst und dann an eine direkt
neben diese Messspule gestellte Arbeitsspule den Befehl zur Beeinflussung sendet.
Anschließend misst es die Zeit, bis ein Wert außerhalb des Rauschens an der Messspule gemessen wird. Dieser Vorgang wird wiederholt durchgeführt.

Die Auswertung der Messwerte hat ergeben, dass die gemessenen Werte zwar variieren, aber immer unter $800 \si{\us}$ und meist auch unter $400 \si{\us}$ liegen. Wenn wir eine maximale Bewegungsgeschwindigkeit des Pendels von $2 \si{m/s}$ annehmen, bedeutet das, dass sich das Pendel nur
$2 \si{m/s} * 400 \si{\us} = 1.6 \si{mm}$ weit von der Spule entfernen kann, bevor die Spule aktiviert wird. Diese Geschwindigkeit sollte mehr als ausreichend hoch sein.

\section{Vorhersage und Kontrolle des Systems}

Der \app{Tracker} kann auch Manipulationen am System über die Arbeitsspulen vornehmen.
Wenn der entsprechende Elektromagnet angeschaltet wird, wenn das Integral einer der Spulen $S_{12}$ oder $S_{13}$ (das sind die untersten beiden Spulen) sein Maximum erreicht hat, wird eine Kraft auf das System ausgeübt, sodass der Magnet verlangsamt wird.
Mit der Zeit bleibt das Pendel so schneller stehen.

Die Vorhersage wird von dem Programm \app{MarkovPrediction} gemacht.
Dazu werden die letzten drei Spulen, an denen der Magnet vorbeigekommen ist, zwischengespeichert.
Wo der Magnet ist, wird vom \app{Tracker} abgefragt.
Für jede neue Spule, an der das Pendel vorbeikommt, wird der Zustand des Systems ermittelt und in der Übergangsmatrix wird die Wahrscheinlichkeit für einen Übergang vom letzten Zustand in den aktuellen Zustand erhöht.
So wird, während Daten eingelesen werden, die Übergangsmatrix immer weiter an die Eigenschaften des Pendels angepasst und die Vorhersage ist \textit{lernfähig}.

Der Zustand besteht aus den letzten drei besuchten Spulen und der Geschwindigkeit des Pendels.
Die Geschwindigkeit kann mithilfe der Zeitstempel und der Position der Spulen ungefähr bestimmt werden.
Diese Information wird in eine Zahl codiert, wobei die Geschwindigkeit zuvor in einen von fünf Bereichen eingeordnet wurde.

\begin{figure}[hbt]
  \centering
  \includegraphics[width=0.7\textwidth]{images/prediction-white.png}
  \caption{Screenshot von der \app{MarkovPrediction}}
  \label{fig:screenmarkov}
\end{figure}

In Abbildung \ref{fig:screenmarkov} ist ein Zustand zu sehen.
Die drei letzten Spulen sind grün markiert und haben die Indizes 14, 15, 7 (der älteste zuerst, die Indizes der Spulen können auch in Abbildung \ref{fig:pendulumsketch} auf Seite \pageref{fig:pendulumsketch} nachgesehen werden).
Die Geschwindigkeit des Pendels beträgt $0,8 m/s$ und wird bei fünf Geschwindigkeitsbereichen und einer Maximalgeschwindigkeit von $1,5 m/s$ in den Bereich mit dem Index zwei eingeordnet:
\mathematik
\textbf{2}\cdot\frac{1,5 m/s}{5} < 0,8 m/s < 3\cdot\frac{1,5 m/s}{5}
\mathematikstop

Alle Geschwindigkeiten größer $1,5 m/s$ werden im obersten Bereich mit dem Index 4 eingeordnet.
Der codierte Index des Zustands ist nun:
\mathematik
2 \cdot 16^3 + 14 \cdot 16^2 + 15 \cdot 16 + 7 = 12023
\mathematikstop

Dieser codierte Index repräsentiert den aktuellen Zustand und stellt den Index in der Übergangsmatrix dar, an dem Wahrscheinlichkeiten für den nächsten Zustand gespeichert werden.
Der wahrscheinlichste nächste Zustand hat den Code 8049, also wird wahrscheinlich die Spule mit einer Geschwindigkeit zwischen $0,3 m/s$ und $0,6 m/s$ an der Spule mit dem Index eins vorbeikommen.
Dies ist auch in der Abbildung als blauer Pfad zu erkennen.

Da es so $16^3 \cdot 5 = 20480$ verschiedene codierte Zustände gibt, ist die Übergangsmatrix relativ groß.
Sie kann in einer Datei gespeichert werden, damit mehr Übergänge eingetragen werden können, nachdem das Programm neu gestartet wurde, und damit sie auch nach einem Neustart weiterhin zur Vorhersage
genutzt werden kann.

Die Genauigkeit der Vorhersage hängt stark von der Bewegung des Pendels ab, da es relativ stark periodische Bewegungen gibt, bei denen das Pendel genau zwischen den Spulen schwingt und daher kein signifikantes Signal erzeugt.
Durchschnittlich kommt die Vorhersage auf $60\%$ Genauigkeit.
In manchen regelmäßigen Bewegungen nähert sich die Genauigkeit $100\%$, aber bei anderen sinkt die Genauigkeit auf unter $20\%$.

Mit der Vorhersage wird es möglich, immer mehr Energie in das System zu bringen.
Wird vorhergesagt, das Pendel sei im nächsten Zustand an Spule $S_{12}$ oder $S_{13}$, kann der entprechende Elektromagnet angeschaltet werden, bevor das Pendel dort ist.
So wird das Pendel beschleunigt, bis die gemessene magnetische Flussdichte ein Maximum erreicht und der Elektromagnet abgeschaltet wird.
Auch hier muss darauf geachtet werden, dass durch das Umschalten der Arbeitsspule mehrere Datensätze verfälscht werden und aussortiert werden müssen.

