Um das Verhalten des Doppelpendels zu beeinflussen, muss der Zustand des Pendels
erfasst werden können. Zu diesem Zweck verwenden wir einen am Ende des Pendels
befestigten Magneten, der in am Rahmen montierten Spulen elektrische Ströme
hervorruft, wenn er an diesen vorbeibewegt wird.

Diese Ströme verursachen Spannungsschwankungen an den Spulen. Um diese zu messen,
verbinden wir alle Spulen auf einer Seite mit einer 3.3V-Spannungsquelle und auf der
anderen mit den 16 Analogeingängen eines ATmega2560-Mikrocontrollers. Der auf einen
Messbereich von 0-5V eingestellte Mikrocontroller iteriert dann in einer Endlosschleife
über alle Eingänge, misst die an ihnen momentan anliegenden Spannungen mit einer
Auflösung von 10 Bit und leitet sie an einen Serial-to-USB-Konverter weiter.

Der Serial-to-USB-Konverter ist per USB an einen Computer angeschlossen, der die 5V-Stromversorgung
bereitstellt und die Messwerte aufzeichnen und verarbeiten kann.

Die 10-Bit-Auflösung des Mikrocontrollers bedeutet, dass die Schrittweite für die Digitalisierung
der Eingangswerte $ 5V / 2^{10} = 4.883mV $ beträgt. Da allerdings bei der Messung ein
Grundrauschen mit einer Breite von ca. XXXX V auftritt, ist die Genauigkeit der Messwerte geringer.

Aus Gründen der Einfachheit - Atmel-Mikrocontroller mit USB-Schnittstelle sind nur als SMD-Chips
verfügbar -, haben wir uns dazu entschlossen, ein Arduino-Mega-2560-Board zu nutzen, das einen
ATmega2560-Mikrocontroller, einen ATmegaXXU2-Mikrocontroller als Serial-to-USB-Converter und eine
3.3V-Spannungsquelle enthält. Das Programm für den ATmega2560, das die Messwerte von den
Analogeingängen liest und über eine serielle Schnittstelle an den ATmegaXXU2 weitergibt, haben
wir aber selbst geschrieben.

Zur Übertragung wird ein Protokoll verwendet, das Messwerte als Kombination aus Eingangs-ID und
Messwert mit hoher Redundanz überträgt, sodass fehlerhafte oder unerwartete Werte abgefangen und
ignoriert werden können.