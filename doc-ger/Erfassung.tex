Um das Verhalten des Doppelpendels zu beeinflussen, muss der Zustand des Pendels
erfasst werden können. Zu diesem Zweck verwenden wir einen am Ende des Pendels
befestigten Magneten, der in am Rahmen montierten Spulen elektrische Ströme
hervorruft, wenn er an diesen vorbeibewegt wird.

Diese Ströme verursachen Spannungsschwankungen an den Spulen. Um diese zu messen,
verbinden wir alle Spulen auf einer Seite mit einer 3.3V-Spannungsquelle und auf der
anderen mit den 16 Analogeingängen eines ATmega2560-Mikrocontrollers. Der auf einen
Messbereich von 0-5V eingestellte Mikrocontroller iteriert dann in einer Endlosschleife
über alle Eingänge, misst die an ihnen momentan anliegenden Spannungen mit einer
Auflösung von 10 Bit und leitet sie an einen Serial-to-USB-Konverter weiter. Der Messbereich
ist dadurch zwar relativ zur Grundspannung asymmetrisch - der maximale negative Ausschlag beträgt
theoretisch 3.3V, während der maximale positive Ausschlag $ 5V - 3.3V = 1.7V $ beträgt -, aber
Spannungsregler für 3.3V sind einfacher verfügbar als welche für 2.5V.
% zusätzliches argument: negative spitzen machen mist?

Der Serial-to-USB-Konverter ist per USB an einen Computer angeschlossen, der die 5V-Stromversorgung
bereitstellt und die Messwerte aufzeichnen und verarbeiten kann.

Die 10-Bit-Auflösung des Mikrocontrollers bedeutet, dass die Schrittweite für die Digitalisierung
der Eingangswerte $ 5V / 2^{10} = 4.883mV $ beträgt. Da allerdings bei der Messung ein
Grundrauschen mit einer Breite von ca. XXXX V auftritt, ist die Genauigkeit der Messwerte geringer.

Aus Gründen der Einfachheit - Atmel-Mikrocontroller mit USB-Schnittstelle sind nur als SMD-Chips
verfügbar -, haben wir uns dazu entschlossen, ein Arduino-Mega-2560-Board zu nutzen, das einen
ATmega2560-Mikrocontroller, einen ATmegaXXU2-Mikrocontroller als Serial-to-USB-Converter und eine
3.3V-Spannungsquelle enthält. Das Programm für den ATmega2560, das die Messwerte von den
Analogeingängen liest und über eine serielle Schnittstelle an den ATmegaXXU2 weitergibt, haben
wir aber selbst in C geschrieben, anstatt die Arduino-spezifische Sprache zu verwenden, die auf
einer höheren Ebene operiert, um möglichst gute Kontrolle über die Geschwindigkeit des Programms
zu haben.

Die Programmierung auf einer solch niedrigen Ebene und das Lesen der Mikrocontroller-Dokumentation
haben gezeigt, dass 500000 Baud die ideale Datenübertragungsrate für die Kommunikation mit dem PC
sind, da bei diesem Wert der Frequenzteiler für die serielle Datenübertragung genau 1 wird.
Tests mit dem in C geschriebenen Programm für den Mikrocontroller haben gezeigt, dass bei
dieser Baudrate die
parallel zur seriellen Datenübertragung ablaufende Analog-Digital-Wandlung die langsamste
Komponente ist, weshalb es ohne Leistungseinbußen möglich ist, für die Übertragung
der Messwerte an den PC ein Protokoll mit hoher Redundanz zu verwenden.

Das Protokoll zwischen Mikrocontroller und PC überträgt Messwerte als Kombination aus
Eingangs-ID und Messwert mit hoher Redundanz, sodass fehlerhafte oder
unerwartete Werte abgefangen und
ignoriert werden können. Solche fehlerhaften Werte sind zwar im laufenden Betrieb relativ selten,
treten aber während der Initialisierung des Arduino häufig auf. Ein weiterer Zweck der
mitgesendeten Eingangs-ID ist es, ohne bidirektionale Kommunikation eine Resynchronisation des
Empfängerprogramms auf dem PC beispielsweise nach einem Neustart zu ermöglichen.