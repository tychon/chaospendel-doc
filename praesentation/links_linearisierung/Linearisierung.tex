\documentclass[a3paper, 12pt, landscape]{article}
\usepackage[utf8]{inputenc}
\usepackage[T1]{fontenc}
\usepackage{ngerman, lmodern}
\usepackage[left=1cm, right=1cm, top=2cm, bottom=2cm]{geometry}

\usepackage{color, amsmath, amsthm, caption}
\usepackage{fancyhdr}
  \lhead{}
  \chead{}
  \rhead{\nouppercase\leftmark}
  \lfoot{}
  \cfoot{\thepage}
  \rfoot{}
\usepackage{multicol}
\columnseprule0.4pt
\columnsep1cm

\usepackage{graphicx}
\newenvironment{Figure}
  {\par\medskip\noindent\minipage{\linewidth}}
  {\endminipage\par\medskip}

\newcommand{\TODO}{\textcolor{red}{ \textbf TODO }}
\newcommand{\mathematik}{\begin{equation*}\begin{aligned}}
\newcommand{\mathematikstop}{\end{aligned}\end{equation*}}
\renewcommand{\phi}{\varphi} % make "stroked" phi look "loopy"
\renewcommand{\rho}{\varrho}
\newcommand{\half}{\frac{1}{2}} % 1/2
\newcommand{\phid}{\dot{\phi}}  % phi with one dot
\newcommand{\phidd}{\ddot{\phi}}  % phi with two dots
\newcommand{\intend}{\,\mathrm{d}} % end of integral
\newcommand{\EQU}{\qquad\bigg|\,} % separator for explanations for equivalent rearrangements

\newcommand{\qtel}{\frac{1}{q}} % separator for explanations for equivalent rearrangements

\title{\vspace{-4cm}\fontsize{96pt}{100pt}\selectfont Linearisierung und Frequenzberechnung}
\author{ }
\date{ }

\begin{document}
\LARGE
\maketitle
\begin{multicols}{2}

Für sehr kleine Auslenkungen $a < 5^\circ$ können die Differentialgleichungen linearisiert werden, indem durch die Annäherungen $cos(\alpha) \approx 1$ und $sin(\alpha) \approx \alpha$ die nicht linearen Anteile eliminiert werden.

\mathematik
\phid_1 &= \frac{k_4 p_1 - k_5 l_1 p_2}{k_1 k_4 + l_1^2 (k_3 k_4 - k_5^2}\\
\phid_2 &= \frac{k_1 p_2 - l_1 (k_3 l_1 p_2 - k_5 p_1)}{k_1 k_4 + l_1^2 (k_3 k_4 - k_5^2}\\
\dot{p}_1 &= -g k_2 \phi_1 - g l_1 k_3 \phi_1 \\
\dot{p}_2 &= -g k_5 \phi_2 \\
\mathematikstop

Nun werden $\phid_1$ und $\phid_2$ noch einmal abgeleitet und $\dot{p}_1$ $\dot{p}_2$ eingesetzt:

\mathematik
\phidd_1 &= \frac{-g (k_2 k_4 \phi_1 + l_1 (k_3 k_4 \phi_1 - k_5^2 \phi_2))}{k_1 k_4 + l_1^2 (k_3 k_4 - k_5^2)} \\
\phidd_1 &= \frac{-g k_5 (k_1 \phi_2 - l_1 (k_2 \phi_1 + k_3 l_1 (\phi_1-\phi_2)))}{k_1 k_4 + l_1^2 (k_3 k_4 - k_5^2)} \\
\mathematikstop

\columnbreak
Mit $q = k_1 k_4 + l_1^2 (k_3 k_4 - k_5^2)$ lassen sich diese Gleichungen so schreiben:

\mathematik
\phidd_1 &= -g k_2 k_4 \phi_1 \qtel - g l_1 k_3 k_4 \phi_1 \qtel + g l_1 k_5^2 \phi_2 \qtel \\
\phidd_2 &= -g k_1 k_5 \phi_2 \qtel + g k_2 k_5 l_1 \phi_1 \qtel + g k_3 k_5 l_1^2 \phi_1 \qtel - g k_3 k_5 l_1^2 \phi_2 \qtel \\
\mathematikstop

So können sie dann in folgende Form gebracht werden. Dies sind die linearisierten Bewegungsgleichungen des Doppelpendels:

\mathematik
\textcolor{red}{\phidd_1} &\textcolor{red}{= a \phi_1 + b \phi_2}\\
\textcolor{red}{\phidd_2} &\textcolor{red}{= c \phi_1 + d \phi_2}\\
\mathematikstop

\mathematik
\textcolor{red}{a}   &\textcolor{red}{= \frac{-g k_4 (k_2 + l_1 k_3)}{k_1 k_4 + l_1^2 \cdot (k_3 k_4 - k_5^2)}}
& \textcolor{red}{b} &\textcolor{red}{= \frac{g k_5^2 l_1}{k_1 k_4 + l_1^2 \cdot (k_3 k_4 - k_5^2)}} \\
\textcolor{red}{c}   &\textcolor{red}{= \frac{g k_5 l_1 (k_2 + k_3 l_1)}{k_1 k_4 + l_1^2 \cdot (k_3 k_4 - k_5^2)}}
& \textcolor{red}{d} &\textcolor{red}{= \frac{-g k_5 (k_1 + k_3 l_1^2)}{k_1 k_4 + l_1^2 \cdot (k_3 k_4 - k_5^2)}} \\
\mathematikstop

\columnbreak
Über den Ansatz

\mathematik
\phi_1 &= A_1 sin\, \omega t
& \phi_2 &= A_2 sin\, \omega t \\
\phidd_1 &= - \omega^2 A_1 sin\, \omega t
& \phidd_2 &= - \omega^2 A_2 sin\, \omega t \\
\mathematikstop

kommt man zu der Matrizengleichungung:

\mathematik
\begin{pmatrix}0 \\ 0\end{pmatrix} &=
\begin{pmatrix}
a+\omega^2 & b\\
c & d+\omega^2
\end{pmatrix}
\begin{pmatrix}A_1 \\ A_2\end{pmatrix}
\mathematikstop

Da diese Gleichung nur lösbar ist, wenn die Determinante der Koeffizentenmatrix Null ist, muss diese quadratische Gleichung gelten:

\mathematik
(a + \omega^2) \cdot (d + \omega^2) - b \cdot c = 0
\mathematikstop

Diese Gleichung hat zwei Lösungen $\omega_1$ und $\omega_2$, die durch $f=\frac{\omega}{2\pi}$ in die Frequenzen umgerechnet werden können.

Die Maße unseres Pendels, und die Koeffizienten mit einer geschätzten Dichtefunktion lauten:
\mathematik
l1a &= -0,1m & l1b &= 0,3m & l1 &= 0,285m \\
l2a &= -0,11m & l2b &= 0,19m \\
\mathematikstop
\mathematik
k1 &= 0,0284 kg\,m^2 \;\; & k2 &= 0,118 kg\,m \\
k3 &= 0,488 kg & k4 &= 0,00441kg\,m^2 \\
k5 &= 0,0254kg\,m
\mathematikstop

Werden diese Wert eingesetzt, kommt man auf die Frequenzen

\mathematik
\textcolor{red}{f_1} &\textcolor{red}{= 0,88Hz} \text{ und}\\
\textcolor{red}{f_2} &\textcolor{red}{= 1,46 Hz}.
\mathematikstop

%\columnbreak\strut % avoid opening of second column
\end{multicols}
\end{document}

